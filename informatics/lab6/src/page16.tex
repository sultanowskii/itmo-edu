\documentclass[main.tex]{subfiles}

\begin{document}
\begin{multicols*}{2}

\noindent\begin{tabular}{
 | >{\Centering}p{0.09\textwidth}
 | >{\Centering}p{0.067\textwidth}
 | >{\Centering}p{0.067\textwidth}
 | >{\Centering}p{0.067\textwidth}
 | >{\Centering}p{0.067\textwidth} | } 
  \hline
  \rowcolor{TableHeader}
  Склады & \multicolumn{4}{c|}{Скупщики} \\
  \cline{2-5}
  \rowcolor{TableHeader}
    & 1 & 2 & 3 & 4 \\
  \hline
  \\[-1em]
  1 & $35^{80}$ & $ ^{120}$ & $ ^{150}$ & $40^{50}$ \\
  \\[-1em]
  \hline
  \\[-1em]
  2 & $25^{60}$ & $10^{70}$ & $40^{90}$ & $^{120}$ \\
  \\[-1em]
  \hline
  \\[-1em]
  3 & $^{120}$ & $50^{50}$ & $^{110}$ & $^{100}$ \\
  \hline
\end{tabular}

\noindent{\smallТабл. 6.}
\medskip

\noindentВ клетках, которые не вошли в цикл, все осталось по-старому.

--- 1400 долларов - кругленькая сумма! Давай проверять другие пустые клетки. Может набредем на маршрут, который тоже стоит использовать. Вот, например, начнем с клетки (1, 2). Для нее расходы изменятся на

$120+60-70-80=30>0$.

\noindentТысяча чертей! Маршрут (1, 2) использовать не стоит. А может быть, воспользоваться...

--- Не трудись, Джо. Я же проверял: больше из этого плана не выжмет ни доллара сам Данциг.

--- Данциг, Данциг... это не тот ли, который обчистил <<Бэнк оф...>>?

--- Нет, Джо, он не из наших. Это тот малый, который придумал этот метод. Правда, еще до него какие-то красные...

... Инспектор Клифф сидел у себя в кабинете на Авеню-стрит, снова и снова всматриваясь в вещественные улики: три мастерски взломанных замка и пепел от тщательно сожженного календаря в гостинице, где совещались грабители. И больше ничего. И все-таки... это напоминает почерк Бэйта, за которым он, Клифф, охотится уже столько лет! К примеру календарь. Зачем он? Может, на нем делались выкладки? Возможно. Но где же искать Бэйта?

--- Сержант! Усиленные наряды во все бары города! - крикнул он, осозновая в то же время полную безнадежность своего приказа: в барах Бэйта не будет. На столе инспектора зазвонил телефон. Клифф снял трубку, послушал и закричал:

--- Сержант, отставить! Оценить научную библиотеку штата! Мне - машину и набор наручников!

\medskip

\noindent\textbf{Немного теории}

\noindentЧто же позволил сэкономить на транспортных расходах 1400 долларов? Проследим за действиями ловких гангстеров. Сначала Бэйт нашел допустимый план перевозок. Метод, которым он при этом воспользовался, называется \textit{методом минимального элемента} и понятно почему: в нем перевозки все время ставятся на маршруты с минимальными тарифами, а если будут два маршрута с одинаковым тарифом, то предпочтение, естественно, нужно отдать тому из них, для которого возможная перевозка больше.

Получив допустимый план, Бэйт и Джо стали пытаться улучшить его \textit{распределительным методом}. Это, пожалуй, самый простой, хотя и не самый быстрый способ улучшения плана перевозок. Но прежде чем излагать этот метод в общем виде, сформулируем строго транспортную задачу \textit{линейного программирования}.

Пусть имеется $m$ поставщиков (складов) и $n$ потребителей, $a_{i}$ - емкость $i$-го склада, а $b_{j}$ - потребность $j$-го потребителя. Пусть $x_{ij}$ - перевозка от $i$-ого поставщика $j$-му потребителю. Допустимы только такие планы перевозок, для которых *)

\setlength{\abovedisplayskip}{0pt}
\setlength{\belowdisplayskip}{0pt}
\noindent\begin{equation}
\begin{split}
\sum_{j=1}^{n} x_{ij} = a_{i}  (i=1,2,...,m), \\
\sum_{i=1}^{m} x_{ij} = b_{j}  (j=1,2,...,n),
\end{split}
\end{equation}

\noindentто есть из каждого склада вывозится все, что там есть, и каждому потребит-

\noindent\rule{0.49\textwidth}{0.1em}

{\fontsize{0.7em}{0.84em}\selectfont*) Мы рассматриваем так называемую <<закрытую>> модель транспортной задачи, для которой $\sum a_{i} = \sum b{j}$, то есть сумма емкостей (складов) равна сумме потребностей.}

\end{multicols*}
\newpage
\end{document}